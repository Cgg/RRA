\part{Techniques de Controle : Reviews}

\section{Revues Techniques}

\begin{description}
\item[Objectifs :]
	\begin{itemize}
	\item détecter des défauts
	\item améliorer la communication
	\item améliorer l'homogénéité du projet
	\item avoir un rôle d'éducation
	\end{itemize}

\item[Types de revues]
	\begin{itemize}

	\item revue technique ( aka revue structurée)
		\begin{itemize}

		\item Informelle
			\begin{itemize}

			\item Cycle Auteur/lecteur (aka Lecture croisée ou peer review)
			\end{itemize}

		\item Formelle
			\begin{itemize}

				\item Revue structurée
				\item Inspection (détails dans la suite)

			\end{itemize}
		\end{itemize}

	\item revue de projet
		\begin{itemize}

		\item Complète
		\item Simplifiée

		\end{itemize}

	\item audit

	\end{itemize}

\end{description}


\section{Inspections ou Walk-Throughs}

\textbf{Idée principale :} Etape de validation de ce qui a été fait. Impossible de passer cette étape si non respect des exigences posées.\\

Le schéma ci dessous représente comment doit se faire la validation d’une inspection. Si inspection échoue, on itère tant qu’on ne la valide pas.\\

Dans ce schéma, on se positionne à l’étape de conception fonctionnelle. Cela dit, la méthode reste valable pour toutes les autres étape de développement d’un logiciel.\\

Définitions des objets utilisés dans le schéma :
\begin{description}
\item[Liste de contrôle] : liste de tous les points où il pourrait y avoir accrochage : Toutes les constantes sont elles bien définies ?
\item [Critères de Sortie] : Liste de critères pouvant faire que l’inspection soit considérée comme un échec
	\begin{itemize}
	\item non prise en compte d'une spécification
	\item non prise  en compte d'une contrainte
	\end{itemize}
\item[Critères d’entré (non représentés)] Ces critères permettent de déterminer si le projet est dans un état tel qu’une inspection est envisageable.
\end{description}


\section{Comment organiser les réunion de revues structurée}

\begin{tabular}{|c|c|c|c|}
\hline
 x && Préparation && Réunion && Suivi
\hline
Durée approximative && 5 jours && 2 heures && 5 jours
\hline
Activité de l’auteur && Fournir le document && Présenter le document, Détailler les points posant problèmes && Corriger le document
\hline 
Activités du président && Designer les participants à la reunion && Diriger les débats && S’assurer que les corrections sont effectuées
\hline 
Activité des participants (sauf auteur)	&& Lire de Document, Corriger les erreur mineures, Définir les points posant problemes && Se mettre d’accord sur les points à corriger &&
\hline
\end{tabular}


\section{Règles de bonnes pratiques des Inspections}

\begin{itemize}
\item distribuer, à l'avance l'information nécessaire (documents, programmes).
\item 3 personnes minimum, 7 personnes maximum
\item Dans un contexte de grands projet ou projets à risque, au moins 3 personnes ne doivent pas prendre part au développement de cette partie/module/composants etc...
\item Pas plus de 2 heures de réunion
\item Inspection =/= conception, on cherche les erreurs et comment les corriger, on ne reconçoit pas l’intégralité de l’application
\item Effectuer un suivi
\item Ne pas juger l’auteur du document.
\end{itemize}
