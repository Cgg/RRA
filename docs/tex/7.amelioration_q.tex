\part{Chapitre 7 : Techniques d’amélioration de la qualité}

\section{Avant-propos}

3 domaines Management par la Qualité : GdP, Support et Doing
4 disciplines du support : AQ (chap 6), Vérification et Validation (V\&V), Gestion de Configuration (GdC) et Test et Evaluation Indépendante (T\&E)\\
V\&V : s’assure conformité par rapport aux exigences (tous niveaux du développement)\\
GdC : assure visibilité permanente de la production du logiciel (suivi des documents, modules …)\\
T\&E : assure recette client d’un système indépendamment de la production

\section{2ème discipline du support : Vérification \& Validation (V\&V)}

	\subsection{Intro aux techniques de contrôle}

\textbf{contrôle} : à partir d’informations on prend une décision, débouche sur des actions correctrices. à la fois validation (bon produit) et vérification (le produit est fait correctement)
évaluation du produit (validation) et du processus (vérification)\\\hfill\\

3 règles d’or :
\begin{itemize}
\item Définir les vérifications : fixer des référentiels + liste des vérifications
\item Mettre en face de chaque vérification les procédures associées : on fixe le controle (quoi, qui, comment, quand …)
\item Spécifier les outils pour : 
	\begin{itemize}
	\item aider à réaliser les controles (aide saisie, recherche automatisée …) 
	\item gérer les controles (confidentialité, enregistrement, traçabilité …)
	\item exploiter les résultats
	\end{itemize}
\end{itemize}

	\subsection{les controles qualité (ou formels)}

outil d’évaluation de la qualité\\
controler “ce qui est nécessaire” par rapport clauses qualités et nature du produit\\
pourquoi controler : vérifier les écarts entre les dispositions d’assurance qualité définies dans le PAQL et la réalité

	\subsection{les controles techniques (sur le fond)}
\textbf{objectifs} : s’assurer tout au long du processus de fabrication :
\begin{itemize}
\item cohérence entre livrables de l’étape n et (actuelle) et résultats de n-1
\item description des interfaces
\item conformité vis à vis exigences fonctionnelles et non fonctionnelles
\end{itemize}

On fait à la fois validation (bon produit) et vérification (faire le produit correctement).\\
Justification des controles techniques : plus on voit une erreur tôt moins ça coûte !\\
On fout des controles tout au long du cycle de développement (définition, conception, réalisation et exploitation).

	\subsection{controle de situation (audit)}
	\begin{description}
	\item[objectif] : faire point sur situation technique
	\item[quand] : demande d’un client
	\item[qui] : personnes indépendantes du projet
	\item[quoi] : organisation de l’entreprise, état d’avancement du logiciel, processus de conduite de projet
	\item[comment] : identifier écarts entre situation réelle et prévue
(détail des différents types d’audits suivant type de référentiel slide 15)
	\end{description}

	\subsection{mise en oeuvre des controles de qualité}
	(3 slides, 3 slides barrés 16 à 18)\\
	présentation d’un exemple : cycle de vie d’un système, positionnement des revues et modèle d’organisation des équipes

\section{3ème discipline du support : Gestion de configuration (GdC)}

(configuration = version)

	\subsection{intro à la GdC}
	\begin{description}
	\item[configuration] : définie par une liste cohérente de composants dans un certain état. Associée à un certain type d’acteurs (utilisateur, développeur …)
	\item[changement de version] : évolution majeure du logiciel
	\item[révision] : évolution mineure
	\end{description}

	\begin{description}
	\item[que doit-on gérer] : schéma dégueulasse slide 21
je dirais qu’on gère le lien entre les différentes versions temporelles et les fichiers (description statique)
	\item[moyens] : PAQL, chaine de production, Dossier de suivi de configuration (DSC), plan de test et environnement de test
	\end{description}

	\begin{description}
	\item[gestions de différents espaces] : espace de référence contient dernière version cohérente
	\item[espace de développement] : pour le développement !
	\item[espace d’archivage] : contient l’ensemble des configurations de référence
	\end{description}

	\begin{description}
	\item[article de configuration] : plus petites entités gérées dans la GdC, plus petites entités pouvant passer d’un espace à un autre
	\end{description}


	\subsection{apports de la GdC}

chacun (chef de projet, resp qualité, équipe de dvpt, client) y trouve son compte (slide 26 pose 4 questions que chacun se pose).\\
GdC consite à :
	\begin{itemize}
	\item gérer l’ensemble des composants (documents, logiciels d’application, logiciels de base, fiches produites pendant dvpt, matériel)
	\item prendre une assurance contre le désordre en assurant la cohérence de ce qui lui est confié
	\item présenter de manière claire et complète la configuration instantanée du produit et l’état d’accomplissement des spécifs fonctionnelles et techniques (doc, source, exécutables …)
	\end{itemize}

	\subsection{champs d’application de la GdC : configurations de référence (ou référentiels)}

Tous les articles sont en cohérence fonctionnelle et technique
article de configuration représenté par : ses composants, les liens entre ces composants, ses interfaces dans le cadre d’ensembles plus larges.\\
Chaque intervenant doit pouvoir manipuler un ensemble cohérent sans perturber les autres : plusieurs types de configuration : de spécification, tests, dvpt et une de référence (liste des composants approuvés qui servent pour phases ultérieures du projet (les autres configurations)).

	\subsection{comment organiser une GdC}

Démarche :
	\begin{itemize}
	\item Analyse du besoin des différents intervenants (espaces nécessaires, activités de gestion de configuration et leur fréquence, confidentialité ...)
	\item Enumération des types d’articles de configuration et de leurs liens
	\item Mise en place de l’organisation
	\item Mise en place des moyens
	\item Identification des articles de configuration pour chaque type retenu
	\item mise en oeuvre
	\end{itemize}
\hfill\\

Attributs d’un article de configuration : identification, auteur, date de création et dernière modif, état (encours, gelé …), droit d’accès, support …\\
Articles groupés en configurations (ensembles complets et cohérents d’articles logiquement reliés) ces liens servent à analyser impact d’une modif, assurer cohérence des liens après modif.\\
À une configuration on associe un arbre de configuration : noeud = objets de type agrégat, feuille = fichiers avec des liens de composition, génération, référence, applicabilité, instanciation, traçabilité (exemples slide 34 à 36).
\hfill\\

4 axes:
	\begin{description}
	\item[Identifier] : chaque configuration -> description technique des articles à chaque instant de leur cycle de vie
	\item[Maitriser] : évolution de chaque configuration
	\item[Enregistrer] : prise en compte de l’état de chaque configuration
	\item[Contrôler] : audits de configuration fonctionnels (ACF) et physiques (ACP)
	\end{description}
\hfill\\

deux actions peuvent déclencher une gestion des modifications :
	\begin{itemize}
	\item détection d’une anomalie (DA) qui génére un rapport d’anomalie (RA)
	\item demande d’évolution demandée par le client (DE)
	\end{itemize}
	\hfill\\

	\subsection{interaction entre la GdC et la production de logiciel}

GdC : états stables et référencés du logiciel, gestion de la configuration de référence d’un logiciel et ses évolutions
Production de logiciel : structuration et maitrise de la construction du logiciel, phase de construction du logiciel.\\

2 approches différents, un objectif commun : faciliter la vie et meilleure qualité

	\subsection{Conclusion}

C’est une activité de soutien de projet, qui doit être adapté aux car de chaque projet.\\
	\begin{description}
	\item[couts] : définition, mise en place et suivi
	\item[gains] : satisfaction du client, homogénéité des dvpts, organisation de la production, qualité des produits, performance
	\end{description}

\section{4ème discipline du support : Test \& Evaluation Indépendante (T\&E)}

	\subsection{objectif de T\&E}
	\begin{description}	
	\item[objectif] : exprimer en phase d’intégration système un point de vue différent et indépendant des équipes système et dvpt de logiciel
	\item[règle] : la prise en compte des évolutions implique une reprise su processus de dvpt qui doit se situer le plus en amont possible
	\end{description}

	\subsection{principales livraisons}
pas compris slide 51

	\subsection{avis de livraison à l’Intégration \& Validation}

plan type pour un avis de livraison :
\begin{description}
\item[identification du logiciel :]
	\begin{itemize}
	\item éléments associés
	\item RA corrigés et non corrigés
	\item DM prises en compte
	\end{itemize}

\item[contenu de la livraison :]
	\begin{itemize}
	\item les PROM
	\item sous produits logiciel
	\item sous systèmes
	\end{itemize}

\item[contenu du logiciel livré :]
	\begin{itemize}
	\item fonctions testées
	\item fonctions non testées
	\item restrictions
	\item infos sur le logiciel livré
	\end{itemize}

\item[procédure d’installation du logiciel]

\item[procédure de lancement du logiciel]

\item[annexes :]
	\begin{itemize}
	\item référenciel prévisionnel
	\item fichiers de démarrage
	\item utilisation particulière
	\end{itemize}

\end{description}

\section{Conclusion}

Toutes ces activités doivent être mises en oeuvre pour obtenir un produit de qualité. Elles ne sont pas indépendantes. Il faut les réduire au maximum $
rightarrow$ réduire les couts.
