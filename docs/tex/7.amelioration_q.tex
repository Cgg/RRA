\part{Techniques d’amélioration de la qualité}

Avant-propos

3 domaines Management par la Qualité : GdP, Support et Doing
4 disciplines du support : AQ (chap 6), Vérification et Validation (V&V), Gestion de Configuration (GdC) et Test et Evaluation Indépendante (T&E)
V&V : s’assure conformité par rapport aux exigences (tous niveaux du développement)
GdC : assure visibilité permanente de la production du logiciel (suivi des documents, modules …)
T&E : assure recette client d’un système indépendamment de la production

2ème discipline du support : Vérification & Validation (V&V)
intro aux techniques de contrôle

contrôle : à partir d’informations on prend une décision, débouche sur des actions correctrices. à la fois validation (bon produit) et vérification (le produit est fait correctement)
évaluation du produit (validation) et du processus (vérification)

3 règles d’or :
Définir les vérifications : fixer des référentiels + liste des vérifications
Mettre en face de chaque vérification les procédures associées : on fixe le controle (quoi, qui, comment, quand …)
Spécifier les outils pour : aider à réaliser les controles (aide saisie, recherche 						automatisée …)

					gérer les controles (confidentialité, enregistrement, 							traçabilité …)
					exploiter les résultats
les controles qualité (ou formels)
outil d’évaluation de la qualité
controler “ce qui est nécessaire” par rapport clauses qualités et nature du produit
pourquoi controler : vérifier les écarts entre les dispositions d’assurance qualité définies dans le PAQL et la réalité
les controles techniques (sur le fond)

objectifs : s’assurer tout au long du processus de fabrication :
cohérence entre livrables de l’étape n et (actuelle) et résultats de n-1
description des interfaces
conformité vis à vis exigences fonctionnelles et non fonctionnelles

on fait à la fois validation (bon produit) et vérification (faire le produit correctement)
justification des controles techniques : plus on voit une erreur tot moins ça coute !
on fout des controles tout au long du cycle de développement (définition, conception, réalisation et exploitation)
controle de situation (audit)

objectif : faire point sur situation technique
quand : demande d’un client
qui : personnes indépendantes du projet
quoi : organisation de l’entreprise, état d’avancement du logiciel, processus de conduite de projet
comment : identifier écarts entre situation réelle et prévue
(détail des différents types d’audits suivant type de référentiel slide 15)
mise en oeuvre des controles de qualité

(3 slides, 3 slides barrés 16 à 18)
présentation d’un exemple : cycle de vie d’un système, positionnement des revues et modèle d’organisation des équipes
3ème discipline du support : Gestion de configuration (GdC)

(configuration = version)
intro à la GdC
configuration : définie par une liste cohérente de composants dans un certain état. Associée à un certain type d’acteurs (utilisateur, développeur …)
changement de version = évolution majeure du logiciel
révision = évolution mineure
que doit-on gérer : schéma dégueulasse slide 21
je dirais qu’on gère le lien entre les différentes versions temporelles et les fichiers (description statique)
moyens : PAQL, chaine de production, Dossier de suivi de configuration (DSC), plan de test et environnement de test
gestions de différents espaces : espace de référence contient dernière version cohérente
espace de développement : pour le développement !
espace d’archivage : contient l’ensemble des configurations de référence
article de configuration : plus petites entités gérées dans la GdC, plus petites entités pouvant passer d’un espace à un autre
apports de la GdC

chacun (chef de projet, resp qualité, équipe de dvpt, client) y trouve son compte (slide 26 pose 4 quesrions que chacun se pose)
GdC consite à :
gérer l’ensemble des composants (documents, logiciels d’application, logiciels de base, fiches produites pendant dvpt, matériel)
prendre une assurance contre le désordre en assurant la cohérence de ce qui lui est confié
présenter de manière claire et complète la configuration instantanée du produit et l’état d’accomplissement des spécifs fonctionnelles et techniques (doc, source, exécutables …)

champs d’application de la GdC : configurations de référence (ou référentiels)

tous les articles sont en cohérence fonctionnelle et technique
article de configuration représenté par : ses composants, les liens entre ces composants, ses interfaces dans le cadre d’ensembles plus larges
caque intervenant doit pouvoir manipuler un ensemble cohérent sans perturber les autres : plusieurs types de configuration : de spécification, tests, dvpt et une de référence (liste des composants approuvés qui servent pour phases ultérieures du projet (les autres configurations))
comment organiser une GdC

démarche :
analyse du besoin des différents intervenants (espaces nécessaires, activités de gestion de configuration et leur fréquence, confidentialité ...)
énumération des types d’articles de configuration et de leurs liens
mise en place de l’organisation
mise en place des moyens
identification des articles de configuration pour chaque type retenu
mise en oeuvre


attributs d’un article de configuration : identification, auteur, date de création et dernière modif, état (encours, gelé …), droit d’accès, support …
articles groupés en configurations (ensembles complets et cohérents d’articles logiquement reliés) ces liens servent à analyser impact d’une modif, assurer cohérence des liens après modif
à une configuration on associe un arbre de configuration : noeud = objets de type agrégat, feuille = fichiers avec des liens de composition, génération, référence, applicabilité, instanciation, traçabilité (exemples slide 34 à 36)

4 axes:
identifier : chaque configuration -> description technique des articles à chaque instant de leur cycle de vie
maitriser : évolution de chaque configuration
enregistrer : prise en compte de l’état de chaque configuration
controler : audits de configuration fonctionnels (ACF) et physiques (ACP)


deux actions peuvent déclencher une gestion des modifications :
détaction d’une anomalie (DA) qui génére un rapport d’anomalie (RA)
demande d’évolution demandée par le client (DE)

interaction entre la GdC et la production de logiciel

GdC : états stables et référencés du logiciel, gestion de la configuration de référence d’un logiciel et ses évolutions
Production de logiciel : structuration et maitrise de la construction du logiciel, phase de construction du logiciel

2 approches différents, un objectif commun : faciliter la vie et meilleure qualité
ccl

C’est une activité de soutien de projet, qui doit être adapté aux car de chaque projet.
couts : définition, mise en place et suivi
gains : satisfaction du client, homogénéité des dvpts, organisation de la production, qualité des produits, performance
4ème discipline du support : Test & Evaluation Indépendante (T&E)
objectif de T&E

objectif : exprimer en phase d’intégration système un point de vue différent et indépendant des équipes système et dvpt de logiciel
règle : la prise en compte des évolutions implique une reprise su processus de dvpt qui doit se situer le plus en amont possible
principales livraisons

pas compris slide 51
avis de livraison à l’Intégration & Validation

plan type pour un avis de livraison :
identification du logiciel :
éléements associés
RA corrigés et non corrigés
DM prises en compte
contenu de la livraison :
les PROM
sous produits logiciel
sous systèmes
contenu du logiciel livré :
fonctions tesstées
fonctions non testées
restrictions
infos sur le logiciel livré
procédure d’installation du logiciel
procédure de lancement du logiciel
annexes :
référenciel prévisionnel
fichiers de démarrage
utilisation particulière

Ccl

Toutes ces activités doivent être mises en oeuvre pour obtenir un produit de qualité. Elles ne sont pas indépendantes. Il faut les réduire au maximum -> réduire les couts.
