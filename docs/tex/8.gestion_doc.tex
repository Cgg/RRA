\part{Chapitre 8 : Gestion de la Documentation}

\section{Introduction}

\begin{description}
\item[Objectif] : Comment gérer la documentation générée par une approche qualité.
\item[Besoins de base] :
	\begin{itemize}
	\item Connaisance de ISO 9000
	\item Exigences de CMMI en matière de documentation (Capability Maturity Model Integration)
	\end{itemize}
\end{description}

Dans toute la suite, on verra défiler les plans des principaux document nécessaires à la gestion documentaire. Il est donc conseillé d’imprimer cette partie plus que de l’apprendre.

\section{Plan type de la documentation}

Sommaire
\begin{enumerate}
\item Contexte
\item Objet
\item Documents applicables et de référence
\item Principaux vocabulaires
\item Formalisation des documents produits
	\begin{enumerate}
	\item Processus de création d'un nouveau document
	\item Présentation des documents
		\begin{enumerate}
		\item " Draft "
		\item Livrables
		\end{enumerate}
	\end{enumerate}
\item Structuration des documents
	\begin{enumerate}
	\item Informations de la page de garde
	\item Infonvations de pied de page
	\item Informations au coeur du document
	\end{enumerate}
\item Gestion des documents produits
	\begin{enumerate}
	\item Etat des documents et gestion des modifications.
	\item Vérification/Validation.
	\item Gestion des versions
	\item Gestion des sauvegardes
	\end{enumerate}
\item Classement des documents produits
	\begin{enumerate}
	\item Journal de la documentation
	\item Nomenclatures et références 
	\item Gestion des répertoires
	\end{enumerate}
\item Gestion de la documentation fournie
	\begin{enumerate}
	\item Documentation informatique
	\item Documentation papier
	\end{enumerate}
\item Gestion du glossaire
	\begin{enumerate}
	\item Définition du glossaire
	\item Acteurs
	\item Processus de remplissage
	\end{enumerate}
\item Plans Type $\rightarrow$ Points importants à définir tôt
\item Annexes
\end{enumerate}

\section{Résumé des Documents nécessaires en fonctions de la taille des projets}

\begin{tabular}{|p{3cm}|p{3cm}|p{3cm}|p{3cm}|}
\hline
 x & Petits projets & Moyens Projets & Fat Projets\\
\hline
Dossier d'init & OUI & OUI & OUI\\
\hline
PAQL & NON & OUI & OUI\\
\hline
Plan de vérification & NON & OUI & OUI\\
\hline
Plan de gestion de configuration & NON & OUI (Si exigence de maintenabilité) & OUI\\
\hline
\end{tabular}
\hfill\\

NB : se référer au chap5 pour le détails de ces documents\\

\textbf{« on n'est pas sûr que l'on fera de la qualité mais, 
on fera au moins de l'assurance »}
