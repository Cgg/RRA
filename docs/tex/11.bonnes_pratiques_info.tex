\part{Les bonnes pratiques en informatique : les référentiels}

\section{Intro}

utilité du référentiel : guide méthodologique efficace et label à mettre en avant

\section{Normes ISO/CEI 12220}

	\subsection{objectif d’un modèle}

constituer cadre de référence\\
structurer organisation du cycle de vie : chronologie temporelle : points de rendez-vous entre les différents acteurs\\
caractériser points d’avacement du projet\\
permettre la gestion de la qualité\\

	\subsection{approche globale "modèle par processus"}

Les modèles par processus ne préconisent pas de méthodes, techniques ou outils particuliers. Ils disent ce qu’il faut faire mais pas comment il faut le faire ni l’organisation à mettre en place, ni de cycle de vie de dvpt.\\
Ils s’orientent progressivement vers l’approche processus généralisée.

	\subsection{la norme ISO 12207}

La norme décrit des exigences pour l’acquisition, la fourniture, le dvpt, la maintenance, l’exploitation des produits logiciels.\\
Elle définit : activités, responsabilités des intervenants et politique de qualité.\\
Elle propose séparation nette entre taches de l’acquéreur et celles du fournisseur, met en place un catalogue des procédures utiles.\\
Elle insiste sur :
\begin{itemize}
\item vérification et validation des exigences
\item traçabilité des exigences
\item anticipation de la maintenance dans phases amont
\item interaction système/logiciel en amont du dvpt
\end{itemize}

\hfill\\

Trois types de processus : métier, de support, organisationnels.
Un processus est découpé en activités. A chaque activité on associe une personnes compétente, une mission, des responsabilités, des liens avec les autres acteurs.

	\subsection{Conclusion}

La norme nécessite un guide de mise en pratique, c’est un référentiel de base.\\
ISO 12207 sert à évaluer le niveau méthodologique d’une entreprise (catalogue de ce qui est maitrisé et des choses à améliorer.

\section{CCMI (Capability Maturity Model Integrated)}

	\subsection{définition}

\begin{itemize}
\item ISO9001 \rightarrow cahier des charges
\item ISO15504 \rightarrow spécifs générales
\item CMMI \rightarrow spécifs détaillées
\item UP (ou USDP) processu unifié
\end{itemize}

	\subsection{historique}

oui mais non

	\subsection{présentation générale}

modèle de bonnes pratiques\\
\textbf{objectif} : améliorer qualité du produit et productivité ;\\ augmenter satisfaction du client ; réduire couts et respecter les délais\\
évaluation de la maturité d’une organisation (de 1 à 5)

	\subsection{méthode de mise en oeuvre}

oui mais non

	\subsection{structure du modèle]

le niveau de maturité est évalué pour les différents domaines de processus. Chaque domaine de processus est constitué d’objectifs spécifiques et d’objectifs génériques, eux-même composés de pratiques spécifiques et génériques.

	\subsection{organisation}

CMMI : “souple sur la forme, dur sur le fond”

	\subsection{représentation étagée}

5 niveaux de maturité :
\begin{enumerate}
\item initial : processus imprévisibles, mal controlés et réactifs
\item processus géré : un processus par projet, parfois réactif
\item processus défini : processus standardisés, gestion proactive
\item processus maitrisé : processus matrisés et controlés
\item optimisation : amélioration continue des processus
\end{enumerate}

OU

\begin{enumerate}
\item initial : processus peu prédictible, peu controlé et réactif
\item reproductible : stucturation et discipline
\item défini : standardisation et institutionnalisation
\item géré quantitativement : mesure prévision
\item optimisé : changement innovation
\end{enumerate}

\section{ITIL (Information Technology Infrastructure Library)}

définition :
\begin{itemize}
\item ensemble cohérent des meilleures pratiques en matière gestion de services info
\item modèle pour la production informatique
\item approche par la définition de processus
\item indépendant des technologies et de l’organisation
\end{itemize}

\hfill\\

objectif :
\begin{itemize}
\item aligner SI avec les objectifs métier
\item améliorer qualité des services fournis
\item maitriser couts
\item mettre en oeuvre une logique de service
\item faire adhérer à ces principes et pas les imposer
\item renforcer la notion de "gestion de projets"
\end{itemize}
