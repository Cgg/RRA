\part{Chapitre 4 : Outils pour la résolution de problèmes et la recherche de la qualité}

\section{Introduction : Qu’est-ce qu’un problème ?}

\begin{description}
	\item[Approche déductive] : Utilisée pour résoudre des problèmes en appliquant des principes physique, chimiques etc... recherche fondamentale.
	\item[Approche inductive] Résolution d’un problème par l'approche qualité :
	\begin{itemize}
		\item Approche et amélioration continue
		\item Approche rationnelle
	\end{itemize}
\end{description}

\section{Quelles démarches pour la qualité ?}

\begin{itemize}
	\item Démarche de l’Analyse Client-Fournisseurs (A.C.F)
	\begin{itemize}
		\item Formaliser les relations client-fournisseur et déterminer un ensemble d'actions pour uniformiser et péréniser cette relation
		\item 10 Phases...
		\item  Maitrise des outils et des compétences de l’équipe !
	\end{itemize}


	\item Le cycle d’amélioration des processus ou process itératifs de la qualité
	\begin{itemize}
		\item Idem que précédemment en focalisant l’approche sur les process uniquement !
	\end{itemize}


	\item Démarche de l’Analyse de la valeur
	\begin{itemize}
		\item AFNOR : Analyse de la méthode visant à la satisfaction du besoin utilisateur par une démarche spécifique de conception.
		\item Faire l’accord entre les coûts et la démarche qualité, l’importance est de bien déterminer la valeur de vente, puis la valeur de fabrication. Adapter le produit aux exigences...
		\item En informatique : On l’applique au niveau du cahier des charges fonctionnel
		\item Il faut ordonnancer par importance les fonctions de base, et investir au maximum sur les exigences importantes, par rapport à ce qui est réellement nécessaire. Quitte à mettre entre parenthèses voire supprimer certains exigences moins prioritaires.
		\item Notion de coût : Traquer les fonctions inutiles, éliminer la surqualité ou l’exploit technique (on s’en fout de la technique \#trollinsa ),
			% Celui-ci, ok, je le laisse \o/ %
	\end{itemize}

\end{itemize}

   
\textbf{TODO Terminer, j’ai du m’endormir à ce moment là :-D}
%TODO%
