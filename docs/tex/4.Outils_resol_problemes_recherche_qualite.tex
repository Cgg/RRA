\part{Chapitre 4 : Outils pour la résolution de problèmes et la recherche de la qualité}

\section{Introduction : Qu’est-ce qu’un problème ?}

Approche déductive : Utilisée pour résoudre des problèmes en appliquant des principes physique, chimiques etc... recherche fondamentale.
Approche inductive :

Résolution d’un problème par l’approche qualité :

Approche et amélioration continue
Approche rationnelle

\section{Quelle démarches pour la qualité ?}

\begin{itemize}
	\item Démarche de l’Analyse Client-Fournisseurs (A.C.F)
	\begin{itemize}
		\item Formaliser les relations client-fournisseur et déterminer un essembles d’action pour uniformiser et péréniser cette relation
		\item 10 Phases...
		\item  Maitrise des outils et des compétences de l’équipe !
	\end{itemize}


	\item Le cycle d’amélioration des processus ou process itératif de la qualité
	\begin{itemize}
		\item Idem que précédemment en focalisant l’approche sur les process uniquement !
	\end{itemize}


	\item Démarche de l’Analyse de la valeur
	\begin{itemize}
		\item AFNOR : Analyse de la méthode visant à la satisfaction du besoin utilisateur par une démarche spécifique du de conception.
		\item Faire l’accord entre les coûts et la démarche qualité, l’importance est de bien déterminer la valeur de vente, puis la valeur de fabrication. Adapter le produit aux exigence ...
		\item En info : On l’applique au niveau du cdcf
		\item Il faut ordonnancer par importance les fonctions de bases, et investir au maximum sur les exigences importantes, par rapport à ce qui est réellement nécessaire. Quitte à mettre entre parenthèses voire supprimer
		\item Notion de coût : Traquer les fonctions inutiles, jarter la surqualité ou l’exploit technique (on s’en fout de la technique \#trollinsa ),
	\end{itemize}

\end{itemize}

   
\textbf{TODO Terminer, j’ai du m’endormir à ce moment la :-D}
