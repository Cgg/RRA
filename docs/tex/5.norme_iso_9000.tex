\part{La norme ISO - 9000}

La qualité repose sur la capacité des fournisseur à maîtriser la qualité, il est donc nécessaire pour les clients de pouvoir évaluer cette qualité. et cela se fait par une grille d’évaluation des systèmes de qualité du fournisseur. Pour que ce système de qualité soit pris au sérieux on lui attribue un nom dégueulasse qui se base sur des nombres à quatre chiffres au détriment de la poésie, de l’amour et de la nature.

Les \textbf{objectifs} de la norme sont de formaliser le système de qualité, de le documenter et de permettre des  actions correctives. il s’agit d’une norme orientée \textbf{processus} et non orientée produit. On ne cherche pas à garantir la qualité, mais bien \textbf{l’application d’un système de qualité}.

La norme ISO-900 apporte une démonstration de \textbf{savoir-faire}, un gain en \textbf{image} et en \textbf{confiance} et privilégie les phases créatrices par rapport aux phases de test et de maintenance en aidant à faire un \textbf{meilleur travail}. En interne la norme aide aussi à assurer la pérennité du savoir-faire (\textbf{capitalisation}) et le management/contrôle financier. La norme recherche \textbf{l’efficience} (l'efficacité au juste coût).

\section{Les 4 verbes de la norme}
\begin{description}
	\item[Plan] Préparer planifier, prévoir
	\item[Do] Réaliser, faire, produire
	\item[Check] contrôler, vérifier, prouver
	\item[Act] Analyser, améliorer, progresser
\end{description}



\section{Approche processus}
\begin{description}
\item[Définition]C’est un ensemble d'activités corrélées ou interactives transformant des éléments d'entrée en éléments de sortie.
\item[Description d’un processus] carte d'identité, carnet de santé, moyens de maîtrise.
\textbf{TODO ICI SCHEMA}
\item[indicateur de conformité] service réalisé / service voulu.
\item[indicateur de satisfaction] service perçu / service attendu.
\end{description}

\hfill\\

Le \textbf{pilote de processus} (responsable) a pour rôle de collecter les indicateurs, rendre compte de l’efficacité du processus et proposer à la direction des actions d’amélioration du processus.


\subsection{Exigences de la norme}

Les éléments du système de qualité doivent être documentés et démontrables d’une manière compatible avec les exigences du modèle sélectionné.

\subsection{Documents exigés par la norme}
\begin{itemize}
	\item Politique et objectifs qualité
	\item Manuel qualité (périmètre d'application, description des interactions des processus, liste des procédures)
	\item Procédures obligatoires
	\item Documents nécessaires pour la planification, le fonctionnement et la maîtrise des processus
	\item enregistrements
\end{itemize}

\subsection{Enregistrements exigés par la norme}
On garde \textbf{TOUT} (la liste est trop longue pour la présenter ici) ce qui concerne de près ou de loin les analyses, actions, compte-rendus, incidents, évaluations, résultats, validation, identification, étalonnage, conformité, formation des employés, etc.

\begin{itemize}
	\item Sur les processus pilotage & support
	\item Sur les processus métiers
\end{itemize}

\subsection{Plan d’assurance qualité}

\begin{decsription}
	\item[Présentation]( Fiche prestation, contrat, clauses qualité.... )
	\item[Analyse] des objectifs de qualité, et des risques « recherche du compromis optimum »
	\item[Référentiel documentaire] (documents du client, système qualité, normes...)
	\item[Organisation] (rôles et responsabilités des personnes)
	\item[Déroulement] (méthodes, règles, livrables, planning... )
	\item[Configuration] (gestion des matériels, logiciels, environnements, versions … )
\end{description}

\begin{tabular}{|c|c|}
	\hline
	Problèmes courants && Solution ISO 9000
	\hline
	Erreur de management && Responsabilités définies clairement
	Manque de communication && Communication, coopération, encouragées
	Incohérence && Standards, gestion de la cohérence
	Produit "Buggé" (non qualité) && Revue, Vérification, Validation
	Erreurs qui se reproduisent && Actions correctives (en principes diminution)
	Peu ou pas de réflexes préventifs && Actions Préventives
	CLIENTS MECONTENTS && QUALITE et SERVICE
	Faible maîtrise && Planification et Gestion
	Charge de travail && Café
\end{tabular}


\section{Critiques de la norme ISO-9000}
\begin{tabular}{|c|c|}
	\hline
	Critique && Contre-Critique
	\hline
	La norme ne donne pas de détail sur le niveau de maîtrise des activités requis && heu, ouais.
	"Trop de rigorisme" "manque de réactivité" && Faute de l’expert qualité : rigueur != rigorisme 
	Culte de l’écrit && Ok pour la formalisation, non pour la paperasserie
	Normes pour l'industrie manufacturière && Les adaptations ont été faite (Extension aux services)
	Aucune prescription (ni méthodes, ni outils ni pratiques de base) && Permet de faire rentrer vos pratiques




\section{Certification du projet qualité}
\begin{itemize}
	\item Déposer une demande à un organisme habilité
	\item Audit par l’organisme habilité sur le site
	\begin{itemize}
		\item Remise d’un rapport précisant ce qui va, ce qui ne va pas
	\end{itemize}
	\item Maintenir le système de qualité
	\begin{itemize}
		\item Audit suivi une fois par an
		\item Audit de renouvellement une fois tous les trois ans
	\end{itemize}
\end{itemize}

La certification n’est qu’un point d’ancrage et ne doit pas être une fin en soi, \textbf{le but c’est la qualité}.