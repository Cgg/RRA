
\section{Cadre général pour la mise en \oe{}uvre d’une démarche qualité pour la production de logiciel}


L’assurance qualité est chargée d’appliquer à l’échelle d’un produit les directives / standards pré-établis d’un système qualité. Ceci avant (préparation), pendant (réalisation) et après (maintenance).
Il est évident que l’AQ peu définir au fil des projets de nouveaux standards visant à améliorer le SQ, mais attention aux fausses \textsl{best practices} ! \\

Trois grandes entités du système qualité logiciel :\\
\begin{itemize}
\item \textbf{La Gestion de configuration} : Assure la visibilité permanente d’une version du soft ainsi que des composants satellites (documentation, modules)
\item \textbf{Vérification \& Validation}  : S’assure de la conformité par rapport aux besoins a tout niveau du dev.
\item \textbf{Tests \& Evaluation} :  S’assure de la recette au niveau client, et de la maintenance derrière pour y apporter quelque chose.
\end{itemize}


Le système qualité modélise et évalue le système de production logiciel en continu. 
Ses objectifs sont de mettre en \oe{}uvre une démarche qualité au niveau Entreprise et au niveau Projet, en gardant toujours un soucis d’optimisation constant.
Chaque projet est unique, et nécessite son propre PAQL !\\

\begin{description}
	\item[Zone d’action d’un Système qualité :]\hfill\\
	\begin{itemize}
		\item \textbf{PdV Organisationnel} : Mise sous contrôle de l’organisation de l’entreprise (ensemble de processus),
		\item \textbf{PdV Démarche} : Approche progressive et évolutive de la démarche, allant de l’expression de la politique qualité, sa mise en place, son entrée dans les moe{}urs et enfin son évaluation.
	\end{itemize}

	\item[Objectifs d’un SQ :]\hfill\\
	\begin{itemize}
		\item Prise en compte du contexte multi-partenaires et les procupéations qualité des différents métiers
		\item Minimiser le risque de dérive des projets
Facilier le retour d’expérience
		\item Maitriser les couts / délais, harmoniser le corps et les normes des procédures
		\item Une visibilité totale du process de production
	\end{itemize}
\end{description}

\section{La structure documentaire d’un système qualité}

\begin{description}
\item[Comment se décompose un SQ :]\hfill\\
\begin{itemize}
\item Structure organisationelle
\item Responsabilités
\item Processus de développement
\item Procédures 
\item Ressources
\end{itemize}
Avec bien entendu la direction au dessus ! \\

\item[Architecture documentaire :]\hfill\\
\begin{itemize}
\item Politique Qualité de l’Entreprise : Manuel Qualité / Manuel Assurance Qualité
\item Procédures : Instructions / Modes opératoires 
\item Projet : Documents (contrats, PAQL, PDL...)
\end{itemize}
\end{description}

	\subsection{Documentation de niveau 1}

\textbf{Manuel Qualité} : énonce et détaille la Politique qualité applicable au niveau de l’entreprise ou d’un sous ensemble sectoriel important. 
Pose les standards de l’entreprise, définition des \textsl{best practices}.\\

\begin{quote}
\em «Avec mon MAQ (MAnuel Qualité) je peux frimer à l’extérieur, et ça se voit à l’interieur !»
\end{quote}
	\subsection{Documentation de niveau 2}

\textbf{Manuel de Mise en \oe{}vre Opérationelle} : énonce les Règles d’ingénierie, outils utilisés, procédures, documentations liées.
Cela cimente le respect des standards et la réutilisation des bests practices, c’est la qualité par construction.

	\subsection{Documentation de niveau 3}

Documentation des projets informatiques (5 grands types): \\
 
\begin{enumerate}
\item Documents de relations Contractuels (CdC, Proposition...)
\item Document de gestion projet (Initialisation, suivi, bilan...)
\item Document d’assurance qualité (PAQL, Audit interne, evaluation en fin de phase, plans de tests) \textbf{Disciplines du Support}
\item Document d’études et de développement (Étude préalable, détaillée, dev\...)
\item Documents d’utilisation (manuel), 
\item Enregistrements qualité (preuves de la mise en \oe{}uvre des règles qualité pendant le projet) $\rightarrow$ comptes-rendus de tests, etc
\end{enumerate}

\hfill\\
Quelques dossiers importants :
\begin{description}
\item[Dossier d’init] Description de la mise en \oe{}uvre des ressources pour développer et produire le produit logiciel en respectant les contraintes. 
Évalue les risques, l’organisation du projet, le planning prévisionel, les moyens mis en \oe{}uvre, documentation de suivi...

\item[PAQL] Modes opératoires, des ressources, des séquences d’activités liés a la qualité. Permet de s’assurer de la mise en \oe{}uvre et de l’efficacité des activités prévues pour obtenir la qualité requise.
\end{description}

\begin{quote}
\textsc Il faut donner confiance au client! 
\end{quote}

\section{1ere discipline du support : l'assurance qualité}

	\subsection{Rôle de l’assurance qualité}

\begin{quote}
«Ensemble de dispositions préétablies et systématiques nécessaires pour donner la confiance appropriée en ce qu’un produit ou service satisfera aux exigences données, relatives à la qualité»
\end{quote}


	\subsection{AQ au niveau de l’entreprise}

Rédaction du Système qualité :\\
	Expérience et culture maison + Contexte + Politique de l’entreprises + Etat de l’art
		$\rightarrow$ CHOIX $\rightarrow$ MAQ, MAQ$_{service}$, spécifications et conceptions\\
			(Recherche des \textsl{best practices})\\

Chaque service d’une entreprise doit rédiger son propre MAQ (MAQL si service informatique) héritant du MAQ de la direction générale.\\

(Exemple sur un cycle en Y et sur un cycle en V de la diapo 25 à 27)\\

Exemple sur la production du code : MAQL :
\begin{itemize}
\item Rédaction et mise en place des procédures
\item Définition de Normes de conception et programmation
\item Rédaction de plan type, règles de productions
\item Choix d’une organisation, des méthodes et des outils
\end{itemize}

(Exemple niveau guide de style diapo 29)

	\subsection{AQ au niveau du projet}

	\begin{description}
		\item[Objectif du génie logiciel] Faire correctement le bon produit dans des conditions satisfaisantes
		\item[Objectif de l’Assurance Qualité Logiciel (AQL)] Bien faire le produit (bonnes règles de prod et bon procédé)
		\item[Contrôle Qualité Logiciel (CQL)] vérifier le produit
		\item[Objectif de la Gestion de projet Logiciel (GdP)] maîtriser le procédé
		\item[RQ] doit s’assurer que les logiciels et processus utilisés sont conformes au exigences du CdC et respectent les plans établis.
	\end{description}
\begin{description}
	\item [Le PAQL doit répondre à :]\hfill\\
    	\begin{description}
    		\item Qui est responsable de quoi ?
    		\item Quels sont les engagements des différents partenaires envers la MOA ?
    		\item Quels sont les Objectifs Qualité du projet ? du produit ?
    		\item Quels sont les circuits d’information, règles de production et de gestion de la documentation ?
    		\item Quels sont les méthodes et outils ?
    	\end{description}

    \item [Portée d’un PAQL] : totale.\\

    \item [Paramètres influençant le PAQL :]\hfill\\
    	\begin{description}
        	\item les exigences du client (cf CdC, éventuellement référentiel d'exigences)
        	\item les difficultés à résoudre le problème
        	\item équipe de développement
        	\item environnement de développement
        	\item stabilité du système
	    \end{description}

    \item [Procédure d’élaboration :]\hfill\\
	    Besoins $\rightarrow$ Classification des exigences Qualité $\rightarrow$ Moyen pour assurer les qualités requises $\rightarrow$ PAQL
\end{description}
	\subsection{Synthèse}
	
\section{Environnement organisationnel de la qualité dans un projet}


Equivalent entre le déroulement du projet et la mise en place du PAQL :\\

\begin{tabular}{|c|p{3cm}|p{3cm}|p{3cm}|p{3cm}|}
\hline
Projet & Avant vente & Lancement du projet & Processus de projet & Clôture du Projet\\
\hline
PAQL & Définition, Rédaction initiale & Rédaction initiale, validation & MAJ Rédaction, validation, mise en \oe{}uvre & Bilan qualité\\
\hline
\end{tabular}

\section{Suivi de la démarche qualité}

Définir des vérifications pour contrôler la conformité, aussi bien sur la forme (structuration des documents, programmes) que sur le fond (conception, réalisation), de l’évolution des processus et des produits par rapport au PAQL $\rightarrow$ rédaction d’un chapitre sur le contrôle de la qualité (cf Chap 12 du PAQL).

\section{Conclusion}

\begin{itemize}
\item Démarche qualité à tous les niveaux (Entreprise, processus, projet/produit)
\item Suivi de la qualité constant
\item Indépendance des documents Qualité
\end{itemize}

\textbf{Prévoir} pour meux \textbf{Agir}, \textbf{Contrôler} pour mieux \textbf{Réagir}.\\

Système Qualité :
\begin{itemize}
\item \textbf{MQ} : Manuel Qualité de l’entreprise
\item \textbf{MAQ} : Manuel d’Assurance Qualité  
\item \textbf{MAQL} : Manuel d'Assurance Qualité Logiciel
\item \textbf{PAQL} : Plan d’Assurance Qualité Logiciel
\item \textbf{PDL} : Plan de Développement de Logiciel 
\item phase de Construction de la Qualité
\item Assurance Qualité
\item Évaluation (Vérification ou Contrôle)
\item Évaluation de la Qualité (META-Qualité)
\end{itemize}

\textbf{La qualité de construit progressivement}\hfill\\

Règles d’or de l’AQL :
\begin{itemize}
\item Découper le processus de développement
\item Concentrer ses efforts sur les points critiques
\item Définir les responsabilités entre partenaires
\item Structurer la documentation
\item Vérifier la qualité en cours de développement
\item Mesurer l’avancement et le comparer avec les prévisions
\item Définir des standards, normes, conventions et \textsl{best practices}
\item Mettre en \oe{}uvre des outils de fabrication
\item Disposer de règles pour maintenir la cohérence du logiciel
\end{itemize}
