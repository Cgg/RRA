\part{Chapitre 2 : le chapitre deux}

\section{Définition de la QUALITE TOTALE}

\begin{description}
\item[Responsable] : Chacun, tous
\item[Comment ameliorer] : PREVENTION
\item[Comment mesurer] : Qualimetres à définir
\item[Niveau de qualité accepté] : Exellence, Zéro defaut
\item[Définition] : Ensemble de Principes et Méthodes, organisés en stratégie globale et visant à mobiliser toute l’entreprise pour obtenir une meilleure satisfaction du client au moindre coût. La \textbf{qualité totale} touche : Tous les employés, tous les services, tous les stades du projet, tous les aspects (qualité, coûts, production …) $\rightarrow$ Exhaustive
\end{description}

\section{Les Objectifs de la démarche QUALITE TOTALE}

	\subsection{Objectif Réel}
Compétitivité : Recherche du moindre coût

	\subsection{Objectif Manifesté}
Zéro défaut

	\subsection{Objectifs Latents}
	\begin{itemize}
	\item Réduire vos coûts inutiles
	\item Accroître votre impact commercial
	\item Renforcer l'efficacité des cadres
	\item Renforcer la motivation du personnel
	\item Améliorer votre organisation 
	\item Raccourcir les délais
	\end{itemize}

	\subsection{Convergence de ces Objectifs}
C’est en faisiant des compromis entre ces objectifs que l’on atteint, étape par étape, la \textbf{qualité totale}.\\ 
Il faut définir : 
	\begin{itemize}
	\item la situation actuelle
	\item les priorités entre les objectifs
	\item les moyens (soit les objectifs intermédiaires)
	\end{itemize}


\section{Processus de mise en oeuvre de la QUALITE TOTALE}

Commencer par :
\begin{itemize}
	\item Audit $\rightarrow$ situation actuelle
	\item Plan d'action progressif sur 5 ans
	\item Ce n’est pas : Tout casser, tout chambouler
	\item C’est : Méthodologie amenée progressivement, en motivant le personnel
\end{itemize}
\hfill\\

Le coût associé à la gestion de la qualité de l'entreprise dépend du DEGRÉ de MATURITÉ de l'entreprise par rapport aux principes de gestion de la QUALITÉ.\\
Cinq degrés de maturité :
\begin{description}
\item[Incertitude]
	\begin{itemize}
	\item pas de conscience des problème de qualité et non-qualité
	\item absence totale de perception du coût des actions de corrections
	\end{itemize}
\item[Mesure (éveil) 6 à 12 mois] : conscient que le travail ne va pas dans le sens de la qualité
\item[Sensibilisation (mise en évidence) 1 an]
	\begin{itemize}
	\item Actions d’évaluation de la qualité systématique
	\item Baisse du coup réel et meilleure évaluation de celui-ci
	\end{itemize}
\item[Formalisation (reconnaissance) 2 à 3 ans] Actions de prévention
\item[Maturité (certitude)] coup perçu = cout réel = coup de prévention + coup d’évaluation
\end{description}
\hfill\\

\textbf{14 étapes selon Crosby}
\begin{enumerate}
\item Engagement de l'encadrement
\item Gestion des programmes d'amélioration de la QUALITÉ
\item Mesure de la QUALITÉ
\item Coût de la QUALITÉ
\item Découverte de la QUALITÉ
\item Actions correctives n n % TODO définir ! %
\item Planification du ZÉRO DEFAUT
\item Formation des cadres 
\item Journées ZÉRO DEFAUT 
\item Fixation des objectifs
\item Élimination des causes d'erreur
\item Reconnaissance des mérites
\item Conseil Qualité
\item Retour à la case départ (Amélioration: mise à jour du Plan d'Action qualité) $\rightarrow$ procéssus itératif
\end{enumerate}


\section{Introduction des principaux concepts d'un Système Qualité}

	\subsection{Vue synthétique des principaux concepts}
TODO : pas compris % TODO ! %

	\subsection{La politique Qualité}
	\begin{itemize}
	\item Slogan simples
	\item Plan d’Amelioration de la Qualité (PAQ)
	\end{itemize}

	\subsection{Maîtrise et appréciation de la Qualité}
Deux aspects de la qualité :
	\begin{itemize}
	\item Qualité du produit
	\item Qualité du processus
	\end{itemize}

	\subsection{Le cycle de la qualité (ou Processus d'amélioration)}
La maîtrise de la qualité se fait sur ces quatre étapes en itérations courtes : 
	\begin{itemize}
	\item Définition des disposition qualité
	\item Élaboration des actions Préventives
	\item Suivi et Évaluation des actions entreprises
	\item Amélioration de l’existant
	\end{itemize}

	\subsection{L'assurance Qualité}
\textit{L'assurance Qualité est définie comme l'ensemble des actions PRE-ETABLIES (Préventives) et SYSTEMATIQUES nécessaires pour donner la confiance appropriée (et A PRIORI) en ce qu'un produit ou service satisfera aux exigences données relatives à la Qualité} (Norme NFX50 120:1997)\\
Deux aspect pour l’assurance qualité
\begin{itemize}
\item Construction $\rightarrow$ dispositions prises
\item Contrôle $\rightarrow$ mesure effectuées
\end{itemize}



	\subsection{La notion de procédure (disposition préventive)}
Mis en oeuvre de l’outil \textbf{QQOQCP} (au nom imprononcable) : Qui Quand Où Quoi Comment Pourquoi
\textbf{TODO} : faire le tri de ce qui est important ou pas

	\subsection{Le système Qualité}
\textit{Le Système Qualité est défini comme l'ensemble de la structure organisationnelle,
des responsabilités, des procédures, des processus (ou procédés) et des moyens
nécessaires pour mettre en oeuvre le management de la Qualité }(NFX50-120)


	\subsection{Le management par la Qualité}
\textbf{TODO}
% TODO %

\section{Différents types de systèmes qualité}

\textbf{Conclusion}
% TODO %
