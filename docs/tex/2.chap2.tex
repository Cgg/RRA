\part{Chapitre 2 : le chapitre deux}

\section{Définition de la QUALITE TOTALE}

\begin{description}
\item[Responsable] : Chacun, tous
\item[Comment ameliorer] : PREVENTION
\item[Comment mesurer] : Qualimetres à définir
\item[Niveau de qualité accepté] : Exellence, Zéro defaut
\item[Définition] : Ensemble de Principes et Méthodes, organisés en stratégie globale et visant à mobiliser toute l’entreprise pour obtenir une meilleure satisfaction du client au moindre coût. La \textbf{qualité totale} touche : Tous les employés, tous les services, tous les stades du projet, tous les aspects(qualité, coûts, production …) $\rightarrow$ Exhaustive
\end{description}

\section{Les Objectifs de la démarche QUALITE TOTALE}

	\subsection{Objectif Réel}
Compétitivité : Recherche du moindre coût

	\subsection{Objectif Manifesté}
Zéro défaut

	\subsection{Objectifs Latents}
	\begin{itemize}
	\item Réduire vos coûts inutile
	\item Accroître votre impact commercial
	\item Renforcer l'efficacité des cadres
	\item Renforcer la motivation du personnel
	\item Améliorer votre organisation 
	\item Raccourcir les délais
	\end{description}

	\subsection{Convergence de ces Objectifs}
C’est en faisiant des compromis entre ces objectifs que l’on atteint, étape par étape, la \textbf{qualité totale}.\\ 
Il faut définir : 
	\begin{itemize}
	\item la situation actuelle
	\item les priorités entre les objectifs
	\item les moyens (soit les objectifs intermédiaires)
	\end{itemize}


\section{Processus de mise en oeuvre de la QUALITE TOTALE}

Commencer par :
\begin{itemize}
\item Audit $\rightarrow$ situation actuelle
\item Plan d’action progressif sur 5 ans
\item Ce n’est pas : Tout casser , tout chambouler
\item C’est : Méthodologie amenée progressivement, en motivant le personel
\end{itemize}
\hfill\\

Le coût associé à la gestion de la qualité de l ' entreprise dépend du DEGRÉ de MATURITÉ de l'entreprise par rapport aux principes de gestion de la QUALITE.\\
Cinq degrés de maturité :
\begin{description}
\item[Incertitude]
	\begin{itemize}
pas de conscience des problème de qualité et nonqualité
Absence totale de perception du coup des actions de corrections
	\end{itemize}
\item[Mesure (éveil) 6 à 12 mois]
conscient que le travail ne vas pas dans le sens de la qualité
\item[Sensibilisation (mise en évidence) 1 an]
Actions d’évaluation de la qualité systématique
Baisse du coup réel et meilleure évaluation de celui-ci
\item[Formalisation (reconnaissance) 2 à 3 ans]
Actions de prévention
\item[Maturité (certitude)]
coup perçu = cout réel = coup de prévention + coup d’évaluation
\end{description}


14 étapes selon Crosby 
Engagement de l'encadrement
Gestion des programmes d'amélioration de la QUALITE
Mesure de la QUALITE
Coût de la QUALITE
Découverte de la QUALITE
Actions correctives n n
Planification du ZERO DEFAUT
Formation des cadres 
Journées ZERO DEFAUT 
Fixation des objectifs
Elimination des causes d'erreur
Reconnaissance des mérites
Conseil Qualité
Retour à la case départ (Amélioration: mise à jour du Plan d'Action qualité)





4 introduction des principaux concepts d'un Système Qualité

4.1 Vue synthétique des principaux concepts
TODO : pas compris

4.2 La politique Qualité
Slogan simples
Plan d’Amelioration de la Qualité (PAQ)

4.4 Maîtrise et appréciation de la Qualité
Deux aspects de la qualité :
Qualité du produit
Qualité du processus

4.5 Le cycle de la qualité (ou Processus d'amélioration)
La maîtrise de la qualité se fait sur ces quatre étapes en itérations courtes : 
Définition des disposition qualité
Élaboration des actions Preventives
Suivi et Evaluation des actions entreprises
Amelioration de l’existant


4.6 L'assurance Qualité
“L'assurance Qualité est définie comme l'ensemble des actions PRE-ETABLIES (Préventives) et SYSTEMATIQUES nécessaires pour donner la confiance appropriée (et A PRIORI) en ce qu'un produit ou service satisfera aux exigences données relatives à la Qualité “ (Norme NFX50 120:1997)
Deux aspect pour l’assurance qualité
Construction -> dispositions prises
Contrôle -> mesure effectuées



4.7 La notion de procédure (disposition préventive)
Mis en oeuvre de l’outil QQOQCP
              ‘-> Qui Quand Où Quoi Comment Pourquoi
TODO : faire le tri de ce qui est important ou pas

4.8 Le système Qualité
“Le Système Qualité est défini comme l'ensemble de la structure organisationnelle,
des responsabilités, des procédures, des processus (ou procédés) et des moyens
nécessaires pour mettre en oeuvre le management de la Qualité “ (NFX50-120)


4.10 Le management par la Qualité
TODO
5 Différents types de systèmes qualité

6 Conclusion
