\chapter*{Avant-propos}
Définition de la qualité (NF ISO 8402) : Ensemble des caractéristiques d’une entité qui lui confèrent l'aptitude à satisfaire des besoins exprimés ou implicites. \\
\begin{itemize}
\item Qualité = politique orientée client. 
\item Le produit doit être adapté aux besoins du client, et non pas être le meilleur produit possible. 
\item Formalisation des méthodes. 
\item Maîtriser les coûts fait partie de la qualité. 
\item La qualité comprend tout, y compris par exemple le marketing et le SAV. 
\item Démarche rigoureuse (et non rigoriste !). 
\item Est-ce que la qualité a un coût ? 
\begin{itemize}
\item Oui car investissement (formation, méthode, réorganisation)
\item Mais selon Ph. CROSBY : \og QUALITY IS FREE \fg =D
\end{itemize}
\item La qualité concerne 3 aspects de l'entreprise : Système qualité (pour mettre en oeuvre le management de la qualité), Conception/Réa (génie logiciel), Produit (sa qualité intrinsèque). 
\item 3 points de vue sur la qualité : utilisateurs, développeurs, décideurs
\item Définitions : 
\begin{itemize}
\item Assurance qualité : Dispositions spécifiques et modalités d'application pour le projet de l'approche qualité globale. 
\item Élaboration de la qualité : Mise en oeuvre des règles, des méthodes, des dispositions qui sont propres à l'entreprise. Buts = détecter et prévenir les non-conformités. 
\item Contrôle qualité : contrôle à des points-clefs, validation de chaque étape (produits intermédiaires). 
\end{itemize}
\end{itemize}

\chapter{Sensibilisation à la démarche qualité}
	\section{Historique} 
		$\rightarrow$ useless
	
	\section{La qualité dans l'entreprise}
		Une démarche qualité impacte sur la structure, l'organisation, les méthodes et la communication de l'entreprise. 
	
	\section{Les 5 principes de la qualité}
		\subsection{Définition/Spécification des exigences}
			\begin{itemize}
			\item réelle difficulté à recueillir les vrais besoins
			\item solution = relation client/fournisseur
			\end{itemize}
		\subsection{Conformité aux exigences}
			\begin{itemize}
			\item c’est le point de départ de la définition de la Q
			\item à chaque transmission entre différents acteurs (client->entreprise, ou différentes équipes au sein de l’entreprise, ex : marketing->cahier des charges), on perd de la qualité. 
			Q souhaitée $>$ Q exprimée $>$ Q interprétée $>$ Q définie $>$ Q conçue $>$ Q réalisée $>$ Q livrée $>$ Q perçue. 
			\item solution = cahier des charges bien défini, V\&V (validation\&vérification), méthodologie...
			\end{itemize}
		\subsection{Prévention plutôt que correction}
			\begin{itemize}
			\item dans le modèle RdP, tout écart entre besoin et résultat = non-conformité
			\item solution = prévoir des manières de travailler pour évaluer les risques et les supprimer (prévention)
			\end{itemize}
		\subsection{Critère de réalisation : le zéro défaut}
			\begin{itemize}
			\item faire bien (conforme aux exigences) du 1er coup et à chaque fois
			\item réalisation de manière progressive, avec des objectifs intermédiaires
			\end{itemize}
		\subsection{Évaluer le coût de la non-conformité}
			\begin{itemize}
			\item non-conformité = 10 à 40\% du prix de revient d’un produit/service
			\item objectif = traquer processus non conformes à l’origine des coûts
			\item évaluation des coûts de tous les processus
			\item coûts de : 
				\begin{itemize}
				\item prévenir/évaluer : COC (coût de la conformité)
				\item faire : coût de conception/réalisation
				\item refaire : CONC (coût de la non-conformité)	
				\end{itemize}
			\end{itemize}
	
	\section{La qualité : qui est concerné ?}
		Tout le monde. \\
		Il faut responsabiliser chaque membre de l’équipe dans la mesure de ses possibilités. \\
		Il faut donner des responsabilités claires. \\
		Il faut évaluer si le travail est bien fait. \\
	
	\section{La qualité : comment ?}
		\begin{itemize}
		\item Clef 1 : Point de vue/état d’esprit qualité
		\item Clef 2 : Mesurer la Q avec des outils 
		\item Clef 3 : La recherche de l’excellence (Plan -> Do -> Check -> Action : cycle universel d’amélioration de la qualité = roue de Deming)
		\end{itemize}

	\section{Quelques outils pour la résolution de problèmes et la recherche de la qualité}
		\subsection{Résolution de problèmes}
			\begin{itemize}
			\item Définition “problème” : écart entre la situation existante et la situation idéale (l’objectif)
			\item 2 approches : déductive/inductive (cc @armandrossius)
			\item La résolution de problèmes par la démarche qualité = approche inductive mais rationnelle : exposer le problème (le comprendre, fixer un objectif, évaluer l'écart entre situation actuelle et objectif) $\rightarrow$ analyser les causes (les rechercher) $\rightarrow$ appliquer des solutions correctives (les définir, les mettre en oeuvre, les institutionnaliser). 
			\item Différence avec méthode classique = analyse des causes au lieu de faire appel à l'intuition et/ou l'expérience. 
			\item La résolution de pb par la démarche qualité utilise : 
				\begin{itemize}
				\item le point de vue qualité
				\item démarche en 7 étapes 
				\item les outils de la qualité
				\end{itemize}
			\item Prérequis pour utiliser cette méthode de résolution : 
				\begin{itemize}
				\item Les raisons qui ont guidé le choix du sujet sont connues, les objectifs sont clairs
				\item Une analyse exhaustive est faite à l'aide d'outils de la qualité
				\item L'analyse a révélé une réelle relation entre causes et effets 
				\item Les solutions sont formulées en faisant preuve d'imagination et d'esprit d'innovation (loul) 
				\item La résolution du problème est abordée du point de vue de la qualité
				\end{itemize}
			\item Les 7 étapes : 
				\begin{itemize}
				\item[1] Choisir un sujet et délimiter précisément le problème
				\item[2] Comprendre la situation, identifier les effets négatifs (par ex : coûts) et fixer les objectifs
				\item[3] Planifier les activités
				\item[4] Analyser les causes majeures du problème
				\item[5] Rechercher les solutions appropriées, élaborer les plans d'action et mettre en oeuvre les solutions
				\item[6] Vérifier les résultats et présenter les conclusions
				\item[7] Définir les indicateurs de contrôle = critères de jugement du résultat (ex : vigueur pour la qualité d'une b*te) 
				\end{itemize}
			\item Résultats attendus : contribution de chacun à la résolution du pb, analyse de tous les aspects du pb, obtention du consensus sur les solutions retenues. 
			\end{itemize}
		\subsection{Quelques méthodes pour la démarche qualité}
			\subsubsection{Démarche de l'Analyse Client-Fournisseur (ACF)}
				\begin{itemize}
				\item Objectif : formaliser les relations Clients-Fournisseurs et déterminer ensemble des actions d'amélioration (qualité du service fourni, fonctionnement interne, meilleure communication entre équipes)
				\item Conditions de réussite : contribution de tout le personnel et travail en groupe, encadrement actif et motivé, volonté absolue d'aboutir, rigueur et sens du concret, organisation structurée. 
				\item Résultats attendus : meilleure communication entre équipes, prise en compte des besoins réels des clients internes et externes, définition conjointe ds indicateurs de mesure. 
				\item Méthode/démarche : en gros, idem que démarche générale en 7 étapes ci-dessus, mais en mettant l'accent sur la recherche des besoins réels du client et la relation client-fournisseur... 
				\end{itemize}
			\subsubsection{Processus itératif de la qualité}
				\begin{itemize}
				\item Objectif : amélioration des processus en s'appuyant plus sur la résolution de pb que sur la hiérarchie et la responsabilité
				\item Démarche : Définir $>$ Mesurer $>$ Analyser $>$ Améliorer $>$ Contrôler
				\item Principe : on fait ça en boucle =D (itération)
				\end{itemize}
			\subsubsection{Démarche de l'analyse de la valeur}
				\begin{itemize}
				\item Définition : L'analyse de la valeur est une méthode de compétitivité organisée et créative visant à la satisfaction du besoin de l'utilisateur par une démarche spécifique de conception à la fois fonctionnelle, économique et pluridisciplinaire. 
				\item Ses principes : valeur, besoins/satisfaction client, méthodologie, collaboration. 
				\item Ses buts : Concevoir au coût le plus faible, satisfaire parfaitement aux besoins, améliorer la qualité, apporter des solutions créatives. 
				\item Démarche : orienté fonctionnalités plutôt que processus $\rightarrow$ Analyse besoins client $>$ Identifier et hiérarchiser les fonctions du produit $>$ Déterminer coût de chaque fonction $>$ Choisir fonction (argumenter) $>$ Valider les fn par rapport au besoin client $>$ Documenter l'étude. 
				\item Résultats attendus : Optimisation des fn et des coûts, adéquation entre solutions retenues et besoins clients (j'ai envie de dire : comme d'hab...). 
				\item Notion de valeur : [service rendu/coût d'acquisition] pour l'utilisateur, ou [valeur de la fn/coût de concept-réa] pour le concepteur. 
				\end{itemize}

		\subsection{Quelques outils de la qualité}
			\subsubsection{Outils de base de 1ère nécessité}
				\paragraph{Diagramme causes-effets}
					\begin{itemize}
					\item Objectif : rechercher systématiquement les causes d'un effet, les classer, et visualiser les liens entre ces causes. 
					\item Méthode : définir l'effet, rechercher les causes, les regrouper en familles, tracer et remplir le diag. de façon claire. 
					\end{itemize}
					[insérer un diagramme ici pour exemple :) p25 du pdf chap1]
				\paragraph{Diagramme de Pareto (80-20)}
					\begin{itemize}
					\item Objectif : déterminer sous forme graphique l'importance relative des différentes causes pour choisir lesquelles traiter. 
					\item Le principe de Pareto : 80\% du problème sont dus à 20\% des causes. 
					\item Démarche : Lister les causes, quantifier leur importance, calculer les pourcentages et tracer le graphique des valeurs cumulées. 
					\end{itemize}
				\paragraph{QQOQCP : Quoi Qui Où Quand Comment Pourquoi ?}	
					\begin{itemize}
					\item QQOQC sont les lignes du tableau, P est la colonne
					\item Objectif : représenter un plan d'action de façon claire et synthétique, pour ne rien oublier. 
					\item TERRAIN ! 
					\end{itemize}
				\paragraph{Remue-méninges (lol) ou Brainstorming}
					\begin{itemize}
					\item Démarche : générer un grand nombre d'idées (réflexion commune) puis les ordonner par famille ou critère puis les exploiter. 
					\item Met en avant la créativité, l'imagination, toussa. 
					\end{itemize}
			\subsubsection{Outils de base de 2ème nécessité}
				\paragraph{OMQ : 7 outils de management de la qualité}
				\begin{itemize}
				\item Mise en oeuvre en 3 phases : clarifier, rechercher et cibler les solutions les plus pertinentes, planifier leur mise en oeuvre
				\item Pour quantifier le qualitatif. 
				\item 7 outils : 
					\begin{itemize}
					\item Diagramme des affinités ou KJ (=QUOI)
					\item Diagramme des relations (=POURQUOI)
					\item Diagramme en arbre (=COMMENT)
					\item Diagramme de décision (=SI ALORS) 
					\item Diagramme en flèches (=QUAND) 
					\item Diagramme matriciel (=LEQUEL)
					\item Analyse factorielle de données
					\end{itemize}
				\end{itemize}
			\subsubsection{Outils de base de 3ème nécessité}
				\begin{itemize}
				\item en complément des 2 premiers si on arrive pas à trouver une solution 
				\item Diagramme polaire (ou Kiviat ou Radar) : ensemble de paramètres sur 1 seul diag. 
				\item Diagramme d'Euler : 3 cercles genre un diag de Venn quoi, pour montrer les combinaisons possibles de 3 éléments. 
				\end{itemize}
					
	
	\section{Approches utilisées pour engager une réflexion qualité}
		\subsection{Approche socio-économique}
		\begin{itemize}
		\item Basée sur changement d'attitudes des individus
		\item Associer performances économiques et performances sociales
		\item Intégrer ttes les catégories de personnel ds une approche structurée et volontariste 
		\item Objectifs : Motiver les personnels, réduire les coûts anormaux,développer les potentiels de chacun, former l'encadrement au management. 
		\end{itemize}
		\subsection{Approche "Démarche de changement"}
			Schéma qu'on ne comprend pas =S c'est juste 'changer pour changer' visiblement
		\subsection{Approche "Élaboration d'un projet d'établissement"}
			Les projets des différents services de l'entreprise constituent le Projet Établissement. 
	
	\section{Conclusion}
		Terrain. 